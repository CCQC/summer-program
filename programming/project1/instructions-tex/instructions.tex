\documentclass[11pt]{article}
\usepackage[cm]{fullpage}
\usepackage{mathtools} %includes amsmath
\usepackage{amsfonts}
\usepackage{listings}
\lstset{basicstyle=\ttfamily\small}
\lstset{literate={~} {$\sim$}{1}}
\lstset{showstringspaces=false}
\usepackage{lmodern}
\usepackage{scrextend}
\usepackage{enumitem}
\usepackage{graphicx}
\usepackage{hyperref}
\usepackage{tikz}
\usepackage{soul}
\usepackage{bm}
\newcommand{\ttf}[1]{{\ttfamily #1}}
\newcommand{\bo}[1]{\ensuremath{\mathbf{#1}}}
\newcommand{\s}{\textbackslash}
\renewcommand{\sp}{\ \ \ \ \ \ \ \ \ \ }
\newcommand{\pt}{\partial}
\newcommand{\fr}[2]{\dfrac{#1}{#2}}
\newcommand{\pr}[1]{\left(#1\right)}
\newcommand{\ord}[1]{\ensuremath{^{(#1)}}}
\newcommand{\ma}[1]{\left(\begin{matrix}#1\end{matrix}\right)}
\newcommand{\ar}[1]{\ensuremath{\begin{matrix} #1 \end{matrix}}}
\newcommand{\pd}[3]{\ensuremath{ \dfrac{ \partial^{#1} #2 }{\partial #3 ^{#1}}}}
\newcommand{\bigo}{\ensuremath{\mathcal{O}}}
\newcommand{\eth}{\ensuremath{^\text{th}}}
\newcommand{\ld}{\ensuremath{\ldots}}
\newcommand{\vd}{\ensuremath{\vdots}}
\newcommand{\dd}{\ensuremath{\ddots}}
\newcommand{\miniar}[1]{\ensuremath{\begin{smallmatrix}#1\end{smallmatrix}}}
\newcommand{\Nu}{\ensuremath{\mathrm{Nu}}}
\newcommand{\op}[1]{\ensuremath{\hat{#1}}}
\newcommand{\Y}{\ensuremath{\Psi}}
\newcommand{\y}{\ensuremath{\psi}}
\newcommand{\si}{\ensuremath{\sigma}}
\newcommand{\w}{\ensuremath{\omega}}
\renewcommand{\d}{\ensuremath{\delta}}
\newcommand{\D}{\ensuremath{\Delta}}
\newcommand{\la}{\ensuremath{\lambda}}
\newcommand{\La}{\ensuremath{\Lambda}}
\newcommand{\tl}[1]{\ensuremath{\tilde{#1}}}
\newcommand{\mf}[1]{\ensuremath{\mathfrak{#1}}}
\newcommand{\e}{\ensuremath{\bm\varepsilon}}

\title{Programming Project 1: Vibrational Frequencies and Normal Modes}
\author{}
\date{}

%beginning of document
\begin{document}
\maketitle

\section*{Description}
The aim of this programming project is to show students how to compute harmonic vibrational frequencies using second derivatives of the potential energy surface.
See section~\ref{background} for details on the theory.

In essence, the task of this project is boils down to two steps:
\begin{enumerate}
	\item formation of the mass-weighted Hessian matrix, which has elements
	\begin{align}
	%
		(\tl{\bo{H}}_0)_{AB}
	=
		\pr{\fr{\pt^2E_e}{\pt\tl{X}_A\pt\tl{X}_B}}_0
	=
		\fr{1}{\sqrt{M_A}}
		\pr{\fr{\pt^2E_e}{\pt X_A\pt X_B}}_0
		\fr{1}{\sqrt{M_B}}
	\end{align}
	where $M_{3A-2}=M_{3A-1}=M_{3A}$ and $X_{3A-2}, X_{3A-1}, X_{3A}$ are the mass and $X,Y,Z$ coordinates of the $A$\eth\ nucleus, and $\tl{X}_A=\sqrt{M_A}X_A$ represents a mass-weighted coordinate.
	\item diagonalization of the mass-weighted Hessian matrix
	\begin{align}
	%
		\tl{\bo{H}}_0
	=
		\bo{L}
		\bm\La
		\bo{L}^T
	\sp
	%
		\bm{\La}
	=
		\ma{\la_1&0&\ld&0\\
			0&\la_2&\ld&0\\
			\vd&\vd&\dd&\vd\\
			0&0&\ld&\la_{3N}}
	\end{align}
where column vectors of the $\bo{L}$ matrix are orthonormal eigenvectors $\bm{l}_1,\ld,\bm{l}_{3N}$ of the mass-weighted Hessian matrix, and $\la_1,\ld,\la_{3N}$ are the corresponding eigenvalues.
\end{enumerate}
In matrix notation, the mass-weighting can be written as
\begin{align}
%
	\tl{\bo{H}}_0
=
	\bo{M}^{-\frac{1}{2}}
	\bo{H}_0
	\bo{M}^{-\frac{1}{2}}
\sp
	\bo{M}
\equiv
	\ma{M_1&0&\ld&0\\
		0&M_2&\ld&0\\
		\vd&\vd&\dd&\vd\\
		0&0&\ld&M_{3N}}
\end{align}
where $\bo{H}_0$ is the Hessian matrix in ordinary Cartesian coordinates.
For eigenvalues $\la_A>0$, $\la_A$ corresponds to the square of the the frequency of molecular vibration
\begin{align}
	\la_A = \w_A^2
\end{align}
along the direction of motion represented by the corresponding eigenvector, which is $\bo{M}^{-\frac{1}{2}}\bm{l}_A$.
This direction of motion corresponds to a {\it vibrational normal coordinate}, typically represented by $Q_A$, and motion along this coordinate corresponds to a displacement of
\begin{align}
%
	\D\bo{X}(Q_A)
=
	Q_A\bo{M}^{-\frac{1}{2}}\bm{l}_A
\end{align}
relative to the reference configuration $\bo{X}_0$.
In addition to the vibrational modes, there will be three {\it translational normal coordinates} for which $\la_A=0$ and three {\it rotational normal coordinates} (two for linear systems) for which $\la_A=0$ holds at an equilibrium geometry.

In order to complete this project you will need the text file containing the Hessian computed in the previous project, as well as the file \ttf{molecule.xyz} and the script you wrote to extract information from that file.
A file called \ttf{masses.py} containing masses for the most stable isotope of every element will also be provided.


\newpage
\section{Procedure}

\begin{enumerate}[label=\textbf{\arabic*}]
%%%%%%%%%%%%
%%%ITEM 1%%%
%%%%%%%%%%%%
\item {\bf Read in the Hessian matrix from a text file as a list of lists of \ttf{float}s. }\\
%%%%%%%%%%%%
%%%ITEM 2%%%
%%%%%%%%%%%%
\item {\bf Form a list containing the diagonal elements of the mass-weighting matrix $\bo{M}^{-\frac{1}{2}}$. }\\
You will first need to get the list of labels from \ttf{molecule.xyz}, which you may want to do by importing your script from the previous project.
Rather than importing the entire script
\begin{addmargin}{2cm}{}
\begin{lstlisting}[language=python]
from scriptname import *
\end{lstlisting}
\end{addmargin}
you should either import the specific function or variable
\begin{addmargin}{2cm}{}
\begin{lstlisting}[language=python]
from scriptname import functionname, variablename
\end{lstlisting}
\end{addmargin}
or import the script as a module
\begin{addmargin}{2cm}{}
\begin{lstlisting}[language=python]
import scriptname
\end{lstlisting}
\end{addmargin}
which allows its functions to be accessed as \ttf{scriptname.functionname()}.
Once you have read in the atom labels, you will be able to form the mass-weighting list
\begin{align*}
	\bo{m}
=
\left[
	\fr{1}{\sqrt{M_1}},
	\fr{1}{\sqrt{M_2}},
	\fr{1}{\sqrt{M_3}},\ld,
	\fr{1}{\sqrt{M_{3N}}} \right]
	\text{\small\ where $M_{3A-2} = M_{3A-1} = M_{3A}$ is the mass of the $A$\eth\ atom}
\end{align*}
using the \ttf{get\_mass()} function in \ttf{masses.py}, which can be called as follows.
\begin{addmargin}{2cm}{}
\begin{lstlisting}[language=python]
>>> import masses
>>> masses.get_mass("C")
12
>>> masses.get_mass("H")
1.00782503207
>>> masses.get_mass("O")
15.99491461956
>>> masses.get_mass("N")
14.00307400478
>>> masses.get_mass("P")
30.973761629
\end{lstlisting}
\end{addmargin}
The square root of a number can be determined either by using the exponential operator in Python \ttf{a**b}~$=a^b$ or by using the square root function \footnote{\url{http://docs.scipy.org/doc/numpy/reference/generated/numpy.sqrt.html}} provided by the NumPy package.
%%%%%%%%%%%%
%%%ITEM 3%%%
%%%%%%%%%%%%
\item {\bf Form $\bo{H}_0$ and $\bo{M}^{-\frac{1}{2}}$ as NumPy matrices.}\\
This can be achieved using the Numpy functions \ttf{numpy.matrix()} \footnote{\url{http://docs.scipy.org/doc/numpy/reference/generated/numpy.matrix.html}} and \ttf{numpy.diag()}.\footnote{\url{http://docs.scipy.org/doc/numpy/reference/generated/numpy.diag.html}}
%%%%%%%%%%%%
%%%ITEM 4%%%
%%%%%%%%%%%%
\item {\bf Form the mass-weighted Hessian $\tl{\bo{H}}_0 = \bo{M}^{-\frac{1}{2}}\bo{H}_0\bo{M}^{-\frac{1}{2}}$.}\\
For NumPy matrices \ttf{A} and \ttf{B}, \ttf{A*B} returns the matrix product.
%%%%%%%%%%%%
%%%ITEM 5%%%
%%%%%%%%%%%%
\item {\bf Determine the eigenvalues and eigenvectors of $\tl{\bo{H}}_0$.}\\
This can be achieved using the \ttf{eigh()} function \footnote{\url{http://docs.scipy.org/doc/numpy/reference/generated/numpy.linalg.eigh.html\#numpy.linalg.eigh}} in the \ttf{numpy.linalg} module.
%%%%%%%%%%%%
%%%ITEM 6%%%
%%%%%%%%%%%%
\item {\bf Use the eigenvalues to determine the vibrational frequencies in cm$^{-1}$ and print them neatly to a file.}\\
Note that wavenumbers are not a true unit of frequency, which physically corresponds to inverse time rather than inverse length.
The ``conversion factor'' between frequencies in wavenumbers (denoted $\tl\nu$) and true frequencies $\nu$ is the speed of light in a vacuum $c$.
\begin{align*}
	\tl\nu = \fr{\nu}{c}
\end{align*}
Be sure that you are aware of the difference between angular frequency (radians/time) and true temporal frequency (cycles/time) before attempting this.
%%%%%%%%%%%%
%%%ITEM 7%%%
%%%%%%%%%%%%
\item {\bf Generate an \ttf{.xyz} file for visualizing the normal modes in Jmol.}\\
\label{last-step}
Note that the Cartesian displacements represented by normal modes correspond to $\bo{M}^{-\frac{1}{2}}\bm{l}_A$ rather than the eigenvectors $\bm{l}_A$ themselves.
All distance units for this file must be in \AA ngstr\"oms, so you will have to apply a unit conversion to both the reference configuration and the mode displacements before printing.
The format of this extended \ttf{.xyz} file used for mode visualization is described below.
\end{enumerate}
Obviously, you should visualize the modes after completing step \ref{last-step}.
If you have Jmol installed, the file generated in that step (call it \ttf{modes.xyz}) can be opened at the command line with the command \ttf{jmol modes.xyz}.
In Jmol, select \ttf{Vibrate...} under \ttf{Tools} and click on \ttf{Start vibration}.
The comment line corresponding to each mode will appear in the lower left corner, and you can cycle through the various modes with the left and right arrow icons.

You should also check your vibrational modes by running a frequency computation in Psi4 with the same method and level of theory.
The simplest way to do this is to copy the input file for your reference geometry computation in the previous project and replace \ttf{energy('scf')} with \ttf{frequencies('scf')}.
You may be surprised at first to find that your rotational normal modes are significantly different from zero.
The reason for this is that the geometry you were given is not an equilibrium structure for H$_2$O at the RHF/cc-pVDZ level, and the eigenvalues corresponding to rotational modes are, in general, nonzero away from equilibrium.
The motivation for starting with a non-equilibrium geometry is that the lifting the degeneracy between rotational and translational motion reduces the numerical error in the diagonalization procedure for determining the corresponding eigenvectors.
As a final step to this project, you should determine the equilibrium geometry for water at the RHF/cc-pVDZ level and compute new frequencies by first running your scripts for determining the Hessian numerically and then running your frequency computation.
You should notice two things:
\begin{enumerate}
	\item you should now have six frequencies near zero (within $50$~cm$^{-1}$ real or imaginary)
	\item your rotational and translational modes will now be barely recognizable when you visualize them in Jmol
\end{enumerate}
After this is done, congratulate yourself and take a break.


\newpage
\section{Extra Files and File Formats}
\subsection{\ttf{masses.py}}
This file provides an easy way to retrieve the mass (in atomic mass units) of the most stable isotope of an element by atomic symbol.
You should look at this file and be sure you understand what it does.
It contains a dictionary \footnote{\url{https://docs.python.org/2/tutorial/datastructures.html\#dictionaries}} to convert from atomic symbol to atomic number as well as a list of isotopic masses (corresponding to the most stable isotope) indexed by atomic number.
The \ttf{get\_mass()} function assumes a string argument, converts the string to uppercase, grabs the corresponding atomic number, and then returns the corresponding mass in the list.


\subsection{\ttf{.xyz} files for visualizing vibrational modes}
The file format for visualizing a single normal mode in Jmol is
\begin{addmargin}{5cm}{}
\begin{lstlisting}[language=c++]
N
COMMENT LINE 1
A1 x1 y1 z1   dx1 dy1 dz1
A2 x2 y2 z2   dx2 dy2 dz2
...                        
AN xN yN zN   dxN dyN dzN
\end{lstlisting}
\end{addmargin}
which amounts to a standard \ttf{.xyz} geometry file (described in the previous handout) with the Cartesian displacement vector for each atom printed next to its Cartesian coordinates.
For visualizing multiple motions, this format is simply repeated with an intervening empty line.
\begin{addmargin}{5cm}{}
\begin{lstlisting}[language=c++]
N
COMMENT LINE 1
A1 x1 y1 z1   dx11 dy11 dz11
A2 x2 y2 z2   dx12 dy12 dz12
...                        
AN xN yN zN   dx1N dy1N dz1N
                           
N                          
COMMENT LINE 2             
A1 x1 y1 z1   dx21 dy21 dz21
A2 x2 y2 z2   dx22 dy22 dz22
...                        
AN xN yN zN   dx2N dy2N dz2N
                           
...                        
                           
N                          
COMMENT LINE N             
A1 x1 y1 z1   dxN1 dyN1 dzN1
A2 x2 y2 z2   dxN2 dyN2 dzN2
...                        
AN xN yN zN   dxNN dyNN dzNN
\end{lstlisting}
\end{addmargin}
Note that, in order to use this file for visualization, the length units must be \AA ngstr\"oms.

\newpage
\section{Background}\label{background}

\subsection{The Born-Oppenheimer Approximation}
Under the Born-Oppenheimer approximation the stationary-states \footnote{\url{http://en.wikipedia.org/wiki/Stationary_state}} of nuclear motion $\Y_\Nu$ arise as solutions of the nuclear motion equation.
\begin{align}
\label{pes}
	\op{H}_\Nu
	\Y_\Nu(\bo{X})
=
	E
	\Y_\Nu(\bo{X})
\sp
	\op{H}_\Nu
\equiv
	\op{T}_\Nu+E_e(\bo{X})
\end{align}
The potential energy term $E_e(\bo{X})$ in the nuclear motion Hamiltonian is defined through the clamped-nuclei Schr\"odinger equation.
\begin{align}
\label{clamped-se}
	\op{H}_e(\bo{X})
	\Y_e(\bo{r}_1,\ld,\bo{r}_n;\bo{X})
=
	E_e(\bo{X})
	\Y_e(\bo{r}_1,\ld,\bo{r}_n;\bo{X})
\end{align}
Equation (\ref{clamped-se}) can be solved approximately at any point $\bo{X}=(\bo{R}_1,\ld,\bo{R}_N)$ using your favorite electronic structure software package (\textsc{Psi4}, CFOUR, Molpro, Gaussian, etc.) and electronic structure method (MP2, CCSD, CISD, etc.).


\subsection{Normal Coordinates and the Vibrational Schr\"odinger Equation}
One of the major barriers to solving the nuclear motion equation (\ref{pes}) is the difficulty of determining the potential energy surface $E_e(\bo{X})$ at a sufficiently large number of points.
Since the nuclei generally remain relatively localized about an equilibrium configuration $\bo{X}_0$, a good first approach is to approximate the potential surface by a Taylor expansion.
\begin{align*}
	E_e(\bo{X})
\approx
	E_e(\bo{X}_0)
+\sum_A^{3N}
	\pr{\pd{}{E_e}{X_A}}_0
	\D X_A
+\fr{1}{2}\sum_{AB}^{3N}
	\pr{\fr{\pt^2E_e}{\pt X_A\pt X_B}}_0
	\D X_A
	\D X_B
	\sp \D\bo{X}=\bo{X}-\bo{X}_0
\end{align*}
Equilibrium structures occur at minima on the potential energy surface, so that the second term is zero and only the quadratic term remains.
\begin{align}
	E_e(\bo{X})
\approx
	E_e(\bo{X}_0)
+\fr{1}{2}\sum_{AB}^{3N}
	\pr{\fr{\pt^2E_e}{\pt X_A\pt X_B}}_0
	\D X_A
	\D X_B
\end{align}
For convenience, we use mass-weighted coordinates $\tl{X}_A \equiv \sqrt{M_A}\D X_A$
in order to remove the explicit dependence of the kinetic energy operator on nuclear masses. \footnote{Explicitly, we can write $\fr{\op{\bo{P}}_A^2}{2M_A} = -\fr{1}{2M_A} \pd{2}{}{X_A}=-\fr{1}{2}\pd{2}{}{(\sqrt{M_A}\D X_A)}=-\fr{1}{2}\pd{2}{}{\tl{X}_A}$. ($\hbar=1$ since we are working in atomic units)}
\begin{align}
	\op{T}_\Nu
=
	-\fr{1}{2}\sum_A^{3N} \pd{2}{}{\tl{X}_A}
\end{align}
The nuclear motion equation (\ref{pes}) then becomes
\begin{align}
\label{pes-intermediate}
\pr{
-\fr{1}{2}\sum_A^{3N}
	\pd{2}{}{\tl{X}_A}
+\fr{1}{2}\sum_{AB}^{3N}
	\pr{\fr{\pt^2E_e}{\pt \tl{X}_A\pt \tl{X}_B}}_0
	\tl{X}_A
	\tl{X}_B
}
	\Y_\Nu(\tl{\bo{X}})
=
	E_\Nu
	\Y_\Nu(\tl{\bo{X}})
\sp
	E_\Nu
\equiv
	E - E_e(\bo{X}_0)
\end{align}
which brings us to the crucial step that motivates this project.

\paragraph{the crucial step that motivates this project:}
Note that, if the mass-weighted Hessian matrix
\begin{align}
	(\tl{\bo{H}}_0)_{AB}
=
	\pr{\fr{\pt^2 E_e}{\pt\tl{X}_A\pt\tl{X}_B}}_0
\end{align}
 were diagonal, we would have a sum of harmonic oscillator Hamiltonians \footnote{The Hamiltonian of a single harmonic oscillator is $\op{H}=\fr{\op{p}^2}{2m}+\fr{m\w^2\op{x}^2}{2}$, or $\op{H}=-\fr{1}{2}\pd{2}{}{q}+\fr{\w^2}{2}q^2$ in mass-weighted coordinates with atomic units.  The frequency of oscillation is $\w$.} in equation (\ref{pes-intermediate}).
This dream of a simple Hamiltonian can be realized by choosing a new set of coordinates in terms of which the Hessian is diagonal.
Since the mass-weighted Hessian $\tl{\bo{H}}_0$ is real and symmetric, it can be diagonalized by a real orthogonal matrix $\bo{L}$ \footnote{\url{http://en.wikipedia.org/wiki/Symmetric_matrix\#Decomposition}}
\begin{align}
	\tl{\bo{H}}_0
=
	\bo{L}\bm{\La}\bo{L}^T
\sp
%
	\bm{\La}
=
	\ma{\la_1&0&\ld&0\\
		0&\la_2&\ld&0\\
		\vd&\vd&\dd&\vd\\
		0&0&\ld&\la_{3N}}
\sp
%
	\bo{L}\bo{L}^T=\bo{L}^T\bo{L}=\bo{1}
\end{align}
where the columns of $\bo{L}$ are eigenvectors $\bm{l}_1,\ld,\bm{l}_{3N}$ of $\tl{\bo{H}}_0$.
The mass-weighted Hessian can be put into this form by expressing our coordinates in the basis of its eigenvectors.
These new coordinates are called {\it normal coordinates}, and the coordinate vector is traditionally referred to as $\bo{Q}$.
The relationship between normal coordinates and mass-weighted Cartesian displacement coordinates is $\bo{Q}=\bo{L}^T\tl{\bo{X}}$, so that the relationship to Cartesian displacements is
\begin{align}
\label{q-to-dx}
%
	\bo{Q}
=
	\bo{L}^T
	\bo{M}^{\frac{1}{2}}
	\D\bo{X}
\sp
%
	\bo{M}^{\frac{1}{2}}
\equiv
	\ma{\sqrt{M_1}&0&\ld&0\\
		0&\sqrt{M_2}&\ld&0\\
		\vd&\vd&\dd&\vd\\
		0&0&\ld&\sqrt{M_{3N}}}
\end{align}
Cartesian displacements therefore depend on the normal coordinates by the following relationship.
\begin{align*}
%
	\D\bo{X}
=
	(\bo{L}^T\bo{M}^{\frac{1}{2}})^{-1}\bo{Q}
=
	\bo{M}^{-\frac{1}{2}}\bo{L}\bo{Q}
\end{align*}
This shows that a motion along the various coordinates $Q_1,\ld,Q_{3N}$ corresponds to a Cartesian displacement by
\begin{align}
%
	\D\bo{X}(Q_1,\ld,Q_{3N})
=
%
	Q_1\pr{\bo{M}^{-\frac{1}{2}}\bm{l}_1}
+
	Q_2\pr{\bo{M}^{-\frac{1}{2}}\bm{l}_2}
+
	\ld
+
	Q_{3N}\pr{\bo{M}^{-\frac{1}{2}}\bm{l}_{3N}}
\end{align}
where the $\bm{l}_A$ represent column vectors of $\bo{L}$, i.e. $\bo{L}=\ma{\bm{l}_1&\ld&\bm{l}_{3N}}$.
We see that each $Q_A$ corresponds to a collective displacement $\D\bo{X}=\bo{M}^{-\frac{1}{2}}\bm{l}_A$ of several nuclei.
The noral coordinates for water are included in an appendix to this handout, and a more detailed interpretation of what these motions represent will be given in the next section.
Note that the Hessian matrix $\bo{H}$ changes continuously across the potential energy surface, so that each point on the surface is associated with its own set of normal coordinates.
Although the normal coordinates at every point span the full space $\mathbb{R}^{3N}$ of nuclear configurations, \footnote{i.e., any set of nuclear positions can be specified by a set of values for $Q_1,\ld,Q_{3N}$} the normal coordinates at a particular point on the potential surface are generally only physically appropriate for describing nuclear motion within a small region about that configuration.

Expressing equation (\ref{pes-intermediate}) in terms of these newfangled coordinates, we are left with 
\begin{align}
\label{vib-se}
\pr{
-\fr{1}{2}\sum_A^{3N}
	\pd{2}{}{Q_A}
+\fr{1}{2}\sum_A^{3N}
	\la_A Q_A^2
}
	\Y_\Nu(\bo{Q})
=
	E_\Nu
	\Y_\Nu(\bo{Q})
\end{align}
(details of the coordinate transformation are shown in an appendix).
The Hamiltonian in (\ref{vib-se}) has the marvelously simple form we were hoping for:
\begin{align}
\label{harm-hamiltonian}
%
	\op{H}_\Nu
\approx
	\sum_{A=1}^{3N}
	\op{\mf{h}}_A
\sp
%
	\op{\mf{h}}_A
\equiv
-\fr{1}{2}
	\pd{2}{}{Q_A}
+\fr{1}{2}
	\la_A Q_A^2
\end{align}
The only approximation for $\op{H}_\Nu$ here is the truncation of the potential energy surface Taylor expansion at second order.
As long as $\la_A$ is a positive real number, $\op{\mf{h}}_A$ is a harmonic oscillator Hamiltonian and we can make the identification $\la_A=\w_A^2$ where $\w_A$ is the oscillator frequency in a.u.
It will turn out that, at an equilibrium geometry, $\tl{\bo{H}}_0$ has several zero eigenvalues -- one for each translational and rotational degree of freedom for the entire nuclear framework with respect to its center of mass.
For these motions, $\op{\mf{h}}_A$ actually takes the form of a free-particle Hamiltonian \footnote{\url{http://en.wikipedia.org/wiki/Free_particle\#Non-Relativistic_Quantum_Free_Particle}}
\begin{align}
%
	\op{\mf{h}}_A
=
-\fr{1}{2}
	\pd{2}{}{Q_A}
\end{align}
although this form is somewhat misleading for rotational motion.
In order to get a clearer picture of what a normal coordinate physically represents, it helps to shift to a classical picture for a moment.

\subsubsection{Classical Interpretation of Normal Coordinate Motions}
Let $\D\bo{X}$ represent a displacement from the equilibrium geometry $\bo{X}_0$ to some new geometry $\bo{X}$:
\begin{align}
	\D\bo{X}
\equiv
	\bo{X}-\bo{X}_0
\end{align}
Then note that $\bo{H}_0\cdot\D\bo{X}$ represents the change in the potential energy surface gradient between those two points
\begin{align*}
%
	\bo{H}_0\cdot\D\bo{X}
=
%
	\nabla E_e
-
	(\nabla E_e)_0
\end{align*}
which can be seen by expanding the gradient at $\bo{X}$ relative to the gradient at $\bo{X}_0$:
\begin{align}
%
	\pd{}{E_e}{X_A}
\approx
%
	\pr{\pd{}{E_e}{X_A}}_0
+\sum_B
	\pr{\pd{}{}{X_B}\pd{}{E_e}{X_A}}_0
	\D X_B
\end{align}
Since $\bo{X}_0$ is an equilibrium geometry, the gradient $(\nabla E_e)_0$ vanishes at that point, and we have simply
\begin{align*}
%
	\bo{H}_0\cdot\D\bo{X}
=
%
	\nabla E_e
\end{align*}
Mass-weighting our coordinate axes leaves the same form of expression.
\begin{align*}
%
	\tl{\bo{H}}_0\cdot\D\tl{\bo{X}}
=
%
	\tl\nabla E_e
\end{align*}
Now, consider what it means for our displacement to be an eigenvector of $\tl{\bo{H}}$.
\begin{align}
%
	\tl{\bo{H}}_0\cdot\D\tl{\bo{X}}
=
%
	\la \D\tl{\bo{X}}
\end{align}
This means that $\D\tl{\bo{X}}$ is proportional to the potential energy surface gradient at the displaced geometry
\begin{align*}
%
	\la
	\D\tl{\bo{X}}
=
%
	\tl{\nabla}E_e
\end{align*}
Explicitly, this can be written as
\begin{align*}
%
	\la
	\sqrt{M_A}
	(X_A-(\bo{X}_0)_A)
=
%
	\pd{}{E_e}{\sqrt{M_A}X_A}
\end{align*}
If we multiply both sides of the equation by $\sqrt{M_A}$, we find that
\begin{align}
%
	\la
	M_A
	(X_A-(\bo{X}_0)_A)
=
%
	\pd{}{E_e}{X_A}
\end{align}
From a classical perspective, the gradient of the potential energy {\small $\pd{}{E_e}{X_A}$} is simply a Cartesian component of (minus) the force \footnote{\url{http://en.wikipedia.org/wiki/Classical_mechanics\#Work_and_energy}} acting on one of the nuclei.
Using Newton's second law, this gives
\begin{align*}
%
	\la
	M_A
	(X_A-(\bo{X}_0)_A)
=
-
	M_A
	(\bo{\ddot{X}})_A
\end{align*}
where $\bo{\ddot{X}}\equiv \fr{d^2\bo{X}}{dt^2}$.
Dividing both sides by $M_A$, we see that the displacement corresponds to a motion parallel to the acceleration
\begin{align}
\label{norm-coords}
-
	\la
	(\bo{X}-\bo{X}_0)
=
%
	\bo{\ddot{X}}
\end{align}
Compare this to Hooke's law for a 1D oscillator, which is $m\ddot{x}=-m\w^2(x-x_0)$.
We see, then, that normal coordinates with positive eigenvalues $\la>0$ correspond to directions of motion along which the potential is ``spring-like'', inducing a force that accelerates the nuclei back to their equilibrium position along the direction of displacement.
It is useful to consider the cases $\la=0$ and $\la<0$ as well.
When $\la=0$, as for translational and rotational modes at equilibrium, there is no force to constrain the motion along these coordinates.
When $\la$ is negative, as will be true of one or more modes at non-equilibrium points, equation (\ref{norm-coords}) shows that displacement induces an acceleration of the nuclei {\it away} from the reference geometry.


\subsubsection{Solutions for Nuclear Motion in a Quadratic Well}
Here I will simply report the quantum mechanical solution to the translational and vibrational components of equation (\ref{vib-se}).
\begin{align*}
\pr{\sum_{A=1}^{3N}
	\op{\mf{h}}_A }
	\Y_\Nu(\bo{Q})
=
	E_\Nu
	\Y_\Nu(\bo{Q})
\sp
	\op{\mf{h}}_A
=
-\fr{1}{2}
	\pd{2}{}{Q_A}
+\fr{1}{2}
	\la_A Q_A^2
\end{align*}
Since the Hamiltonian separates into a sum of independent Hamiltonians each involving only one coordinate, the form of the wavefunction is
\begin{align}
	\Y_\Nu(Q_1,\ld,Q_{3N})
=
	\y_1(Q_1)
	\y_2(Q_2)
	\cdots
	\y_{3N}(Q_{3N})
\end{align}
where each one-coordinate wavefunction $\y_A(Q_A)$ is an eigenfunction (or, rather, {\it one of the} eigenfunctions) of the corresponding operator, $\op{\mf{h}}_A$.
\begin{align}
	\op{\mf{h}}_A
	\y_A(Q_A)
=
	\e
	\y_A(Q_A)
\end{align}
We will only need solutions to the 1D free-particle and 1D harmonic oscillator Schr\"odinger equations for the present discussion, which can be found in any introductory quantum mechanics text worth reading.\footnote{e.g. R.\ Shankar {\it Principles of Quantum Mechanics}, which contains detailed discussions of both the free particle and harmonic oscillator. An electronic copy can be found \href{http://home.basu.ac.ir/\~psu/Books/\%5BRamamurti_Shankar\%5D_Principles_of_Quantum_Mechanic\%28BookFi.org\%29.pdf}{here}.}

For the three normal coordinates that correspond to translations of the molecule as a whole, the potential term in $\op{\mf{h}}$ vanishes (since $\la=0$) and we only need to solve a free-particle-like problem.
\begin{align}
-\fr{1}{2}
	\pd{2}{\y(Q_A)}{Q_A}
=
	\e
	\y(Q_A)
\end{align}
Solutions to this differential equation are given by
\begin{align}
	\y_{P_A}(Q_A)
=
	e^{iP_A(Q_A-\theta_A)}
\end{align}
where $\theta_A$ is some undetermined phase shift.
The distribution of eigenvalues is continuous in this case, $\e_{P_A}=\fr{P_A^2}{2}$, where $P_A$ corresponds to a component of the linear momentum of the molecule as a whole.

Although $\la=0$ also holds for rotational modes at equilibrium, normal coordinates are only appropriate for parametrizing infinitesimal rotations and quickly break down for larger rotational motions (see the images in section \ref{rot}).
Furthermore, unlike translations, these modes are not truly ``external'' since they change the relative positions of the nuclei (and therefore change the energy for finite displacements).
Proper parametrization of a full rotation about the center of mass in terms of fixed normal coordinates would actually require linear combinations of rotational modes (section \ref{rot}) with vibrational ones (section \ref{vib}), and these motions are in general coupled to each other.
For our purposes, we will simply note one can, to a good first approximation, ignore the interference of rotations when considering vibrational motion. \footnote{This briefly discussed in N.\ V.\ Cohan and H.\ F.\ Hameka, {\it J. Chem. Phys.} 45, 4392 (1966).}

The remaining $3N-6$ coordinates ($3N-5$ for linear molecules) will always have $\la>0$ at an equilibrium geometry and so lead to true harmonic oscillator motion.
The solutions to this problem
\begin{align}
-\fr{1}{2}
	\pd{2}{\y(Q_A)}{Q_A}
+\fr{1}{2}
	\la_A
	Q_A^2
=
	\e
	\y(Q_A)
\end{align}
are given by
\begin{align}
%
	\y_{n_A}(Q_A)
=
\fr{1}{\sqrt{2^n n!}}\pr{\fr{\w_A}{\pi}}^{\frac{1}{4}}
	e^{-\frac{1}{2}\w_AQ_A^2}
	\ \text{He}_n(\sqrt{\w_A}Q_A)
\sp
	\w_A^2=\la_A
\end{align}
with energies $\e_{n_A}=\w_A(n_A+\frac{1}{2})$, where $\text{He}_n(x)$ is the $n$\eth\ Hermite polynomial.
\begin{align}
%
	\text{He}_n(x)
=
	(-1)^n
	e^{x^2}
	\pr{\fr{d^n}{dx^n}e^{-x^2}}
\end{align}
Hence, the solution to the {\it vibrational Schr\"odinger equation} \footnote{This is simply equation (\ref{vib-se}) with translational and rotational modes omitted.}
\begin{align}
\sum_{A=1}^{3N-6}
	\pr{-\fr{1}{2}\pd{2}{}{Q_A}+\w_A^2Q_A^2}
	\Y_\text{vib}
=
	E_\text{vib}
	\Y_\text{vib}
\end{align}
is given by
\begin{align}
%
	\Y_{n_1,\ld,n_{3N-6}}(Q_1,\ld,Q_{3N-6})
=
\prod_{A=1}^{3N-6}
	\y_{n_A}(Q_A)
\end{align}
with vibrational energy
\begin{align}
%
	E_{n_1,\ld,n_{3N-6}}
=
\sum_{A=1}^{3N-6}
	\w_A
	\pr{n_A+\fr{1}{2}}
\end{align}
The total molecular energy at equilibrium is the sum of translational, rotational, vibrational, and electronic contributions.
\begin{align}
	E
=
	\fr{\bo{P}_\text{CM}^2}{2M_\text{tot}}
+
	E_\text{vib}
+
	E_\text{rot}
+
	E_e(\bo{X}_0)
\end{align}



\newpage
\subsection{Appendix: Normal Coordinates for Water}
\subsubsection{Translational Modes}
\label{trans}
\begin{figure}[htp]
	\centering
	\begin{tabular}{|ccc|}\hline&&\\
	\includegraphics[width=5.5cm,clip=true,trim=3cm 2cm 3cm 2cm]{0-0_0.pdf}&
	\includegraphics[width=5.5cm,clip=true,trim=3cm 2cm 3cm 2cm]{0-90_0.pdf}&
	\includegraphics[width=5.5cm,clip=true,trim=3cm 2cm 3cm 2cm]{90-0_0.pdf}\\front view&side view&top view\\\hline&&\\
	\includegraphics[width=5.5cm,clip=true,trim=3cm 2cm 3cm 2cm]{0-0_1.pdf}&
	\includegraphics[width=5.5cm,clip=true,trim=3cm 2cm 3cm 2cm]{0-90_1.pdf}&
	\includegraphics[width=5.5cm,clip=true,trim=3cm 2cm 3cm 2cm]{90-0_1.pdf}\\front view&side view&top view\\\hline&&\\
	\includegraphics[width=5.5cm,clip=true,trim=3cm 2cm 3cm 2cm]{0-0_2.pdf}&
	\includegraphics[width=5.5cm,clip=true,trim=3cm 2cm 3cm 2cm]{0-90_2.pdf}&
	\includegraphics[width=5.5cm,clip=true,trim=3cm 2cm 3cm 2cm]{90-0_2.pdf}\\front view&side view&top view\\\hline
	\end{tabular}
\end{figure}
\newpage
\subsubsection{Rotational Modes}
\label{rot}
\begin{figure}[htp]
	\centering
	\begin{tabular}{|ccc|}\hline&&\\
	\includegraphics[width=5.5cm,clip=true,trim=3cm 2cm 3cm 2cm]{0-0_3.pdf}&
	\includegraphics[width=5.5cm,clip=true,trim=3cm 2cm 3cm 2cm]{0-90_3.pdf}&
	\includegraphics[width=5.5cm,clip=true,trim=3cm 2cm 3cm 2cm]{90-0_3.pdf}\\front view&side view&top view\\\hline&&\\
	\includegraphics[width=5.5cm,clip=true,trim=3cm 2cm 3cm 2cm]{0-0_4.pdf}&
	\includegraphics[width=5.5cm,clip=true,trim=3cm 2cm 3cm 2cm]{0-90_4.pdf}&
	\includegraphics[width=5.5cm,clip=true,trim=3cm 2cm 3cm 2cm]{90-0_4.pdf}\\front view&side view&top view\\\hline&&\\
	\includegraphics[width=5.5cm,clip=true,trim=3cm 2cm 3cm 2cm]{0-0_5.pdf}&
	\includegraphics[width=5.5cm,clip=true,trim=3cm 2cm 3cm 2cm]{0-90_5.pdf}&
	\includegraphics[width=5.5cm,clip=true,trim=3cm 2cm 3cm 2cm]{90-0_5.pdf}\\front view&side view&top view\\\hline
	\end{tabular}
\end{figure}
\newpage
\subsubsection{Vibrational Modes}
\label{vib}
\begin{figure}[htp]
	\centering
	\begin{tabular}{|ccc|}\hline&&\\
	\includegraphics[width=5.5cm,clip=true,trim=3cm 2cm 3cm 2cm]{0-0_6.pdf}&
	\includegraphics[width=5.5cm,clip=true,trim=3cm 2cm 3cm 2cm]{0-90_6.pdf}&
	\includegraphics[width=5.5cm,clip=true,trim=3cm 2cm 3cm 2cm]{90-0_6.pdf}\\front view&side view&top view\\\hline&&\\
	\includegraphics[width=5.5cm,clip=true,trim=3cm 2cm 3cm 2cm]{0-0_7.pdf}&
	\includegraphics[width=5.5cm,clip=true,trim=3cm 2cm 3cm 2cm]{0-90_7.pdf}&
	\includegraphics[width=5.5cm,clip=true,trim=3cm 2cm 3cm 2cm]{90-0_7.pdf}\\front view&side view&top view\\\hline&&\\
	\includegraphics[width=5.5cm,clip=true,trim=3cm 2cm 3cm 2cm]{0-0_8.pdf}&
	\includegraphics[width=5.5cm,clip=true,trim=3cm 2cm 3cm 2cm]{0-90_8.pdf}&
	\includegraphics[width=5.5cm,clip=true,trim=3cm 2cm 3cm 2cm]{90-0_8.pdf}\\front view&side view&top view\\\hline
	\end{tabular}
\end{figure}

\newpage
\subsection{Appendix: Details of the Coordinate Transformation}
It is worth showing how this coordinate transformation works explicitly.
In particular, we should show how the derivatives transform to arrive at equation (\ref{vib-se}).
Since $\tl{\bo{X}}^T\tl{\bo{H}}_0\tl{\bo{X}}=(\bo{L}^T\tl{\bo{X}})^T\bm\La\bo{L}^T\tl{\bo{X}}$, the normal coordinates are given in terms of mass-weighted Cartesian displacements as
\begin{align*}
	\bo{Q}
=
	\bo{L}^T\tl{\bo{X}}
\end{align*}
or
\begin{align}
%
	Q_A(\tl{X}_1,\ld,\tl{X}_{3N})
=
\sum_{B=1}^{3N}
	L_{BA}
	\tl{X}_B
\end{align}
Then the derivative with respect to $\tl{X}_A$ can be expressed in terms of derivatives with respect to normal coordinates $\{Q_A\}$ as
\begin{align}
\label{derivative-transform}
%
	\pd{}{}{\tl{X}_A}
=&\
\sum_B
	\pd{}{Q_B}{\tl{X}_A}\pd{}{}{Q_B}
=
\sum_{BC}
	\pd{}{L_{CB}\tl{X}_C}{\tl{X}_A}\pd{}{}{Q_B}
=
\sum_{B}
	L_{AB}\pd{}{}{Q_B}
\end{align}
where we have used the fact that
\begin{align*}
\pd{}{\tl{X}_C}{\tl{X}_A}=\d_{CA}
\end{align*}
($\d_{CA}$ being the Kronecker delta \footnote{\url{http://en.wikipedia.org/wiki/Kronecker_delta}}).
Then we can see that the kinetic energy operator transforms as follows.
\begin{align*}
\sum_A
	\pd{2}{}{\tl{X}_A}
=
\sum_{ABC}
	\fr{\pt^2}{\pt Q_B\pt Q_C}
	L_{AB}L_{AC}
=
\sum_{BC}
	\fr{\pt^2}{\pt Q_B\pt Q_C}
	(\bo{L}^T\bo{L})_{BC}
\end{align*}
Since $\bo{L}$ is an orthogonal matrix, $(\bo{L}^T\bo{L})_{BC}=(\bo{1})_{BC}=\d_{BC}$, and so
\begin{align}
%
	\op{T}_\Nu
=
-\fr{1}{2}\sum_A
	\pd{2}{}{\tl{X}_A}
=
-\fr{1}{2}\sum_B
	\pd{2}{}{Q_B} \ \ .
\end{align}
In a similar fashion, we can verify that the transformed Hessian matrix is in fact simply the Hessian with respect to normal coordinates by comparing
\begin{align*}
%
	\pr{\fr{\pt^2E_e}{\pt\tl{X}_A\pt\tl{X}_B}}_0
=
\sum_{CD}
	\pr{\fr{\pt^2E_e}{\pt Q_C\pt Q_D}}_0
	L_{AC}L_{BD}
\end{align*}
to
\begin{align*}
%
	\pr{\fr{\pt^2E_e}{\pt\tl{X}_A\pt\tl{X}_B}}_0
=
	(\tl{\bo{H}}_0)_{AB}
=
	(\bo{L}\bm\La\bo{L}^T)_{AB}
=
	\sum_{CD}
	(\la_C\d_{CD})L_{AC}L_{BD}
\end{align*}
which shows that
\begin{align}
%
	\pr{\fr{\pt^2E_e}{\pt Q_C\pt Q_D}}_0
=
	\la_C\d_{CD}
\end{align}
That is, we have identified the diagonal elements $\{\la_A\}$ of $\bm\La$ as corresponding to
\begin{align}
	\la_A \equiv \pr{\pd{2}{E_e}{Q_A}}_0
\end{align}
and shown that mixed derivatives are zero
\begin{align}
	\pr{\fr{\pt^2E_e}{\pt Q_A\pt Q_B}}_0
=
	0
\sp
	\text{ if } A\neq B \ .
\end{align}






%end of the document
\end{document}
