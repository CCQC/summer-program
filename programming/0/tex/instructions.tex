\documentclass[11pt]{article}
\usepackage{mystyle}

\title{Do you want to build a molecule?}
\date{}
\begin{document}
\maketitle

This programming project will teach you the basics of how to properly implement
a simple molecule class. In doing so, you will learn how to program in Python
as well as proper coding techniques.

To begin, will will first need to make a Python molecule class

\linp[firstline=3, lastline=7]{../jevandezande/molecule.py}

This tells us that we have a new class called Molecule, and the docstring (that
thing in between the triple quotes) gives us a little information about it. Next
we need to write a function to be run every time we create a new molecule.

\linp[firstline=9, lastline=16]{../jevandezande/molecule.py}

The function \linl{__init__} is run every time a new molecule is created.
It takes three arguments, \linl{self} (which is a standin for the name of
the new object, such as mol in \linl{mol = Molecule(geom_str)}),
\linl{geom_str} (which is a string of the geometry), and units (which
default to Angstrom). The function then reads the geometry string and sets the
units. Thus we need to write a function to actually read the geometry and save
it in a usable form. Here is a stub detailing how it should be implemented.

\linp[firstline=35, lastline=44]{../jevandezande/molecule.py}

Now that we have saved out geometry, it would be nice to be able to get various
different things about our molecule. There are some built-in functions in Python
that you may have run across. \linl{len(mylist)} will return you the
length of \linl{mylist}. I would recommend glancing over all the
available built-in functions (https://docs.python.org/3/library/functions.htm).
We can override the function \linl{len} do allows us to get the length
of our molecule, here is how to start.

\linp[firstline=29, lastline=32]{../jevandezande/molecule.py}

Once you have figured how to return the length of the molecule, we need to make
a nice way to print it. To do so, we will need to overide the built-in function
\linl{str}. We will want it to look just like the original xyz file. Here is some
starter code, you will need to review the string format function to determine
how to use it.

\linp[firstline=18, lastline=23]{../jevandezande/molecule.py}

Finally, we would like to be able to convert from Angstrom to Bohr, and back.
This can be achieved by multiplying or dividing, respectively, by 1.889725989
each of the coordinates. Since we are using a numpy array, this is very easy,
but I will leave it to you to figure out how to accomplish this in one line
for each function.

Now that you are done, look at my code, available in the GitHub repo under
jevandezande. How does yours compare, see if you can understand some of the
tricks I used to simplify my code.

\end{document}
