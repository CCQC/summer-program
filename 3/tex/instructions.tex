\documentclass[11pt]{article}

\usepackage{listings}
\lstset{basicstyle=\ttfamily\small}
\lstset{literate={~} {$\sim$}{1}}
\lstset{showstringspaces=false}
\lstset{language=Python}
\usepackage{scrextend}
\newcommand{\linp}[1]{\lstinputlisting{#1}{}}
\newcommand{\linl}[1]{\lstinline{#1}{}}

\usepackage[cm]{fullpage}
\usepackage{mathtools} %includes amsmath
\usepackage{amsfonts}
\usepackage{bm}
\usepackage{url}
%greek letters
\renewcommand{\a}{\alpha}    %alpha
\renewcommand{\b}{\beta}     %beta
\newcommand{\g}{\gamma}      %gamma
\newcommand{\G}{\Gamma}      %Gamma
\renewcommand{\d}{\delta}    %delta
\newcommand{\D}{\Delta}      %Delta
\newcommand{\e}{\varepsilon} %epsilon
\newcommand{\ev}{\epsilon}   %epsilon*
\newcommand{\z}{\zeta}       %zeta
\newcommand{\h}{\eta}        %eta
\renewcommand{\th}{\theta}   %theta
\newcommand{\Th}{\Theta}     %Theta
\newcommand{\io}{\iota}      %iota
\renewcommand{\k}{\kappa}    %kappa
\newcommand{\la}{\lambda}    %lambda
\newcommand{\La}{\Lambda}    %Lambda
\newcommand{\m}{\mu}         %mu
\newcommand{\n}{\nu}         %nu %xi %Xi %pi %Pi
\newcommand{\p}{\rho}        %rho
\newcommand{\si}{\sigma}     %sigma
\newcommand{\siv}{\varsigma} %sigma*
\newcommand{\Si}{\Sigma}     %Sigma
\renewcommand{\t}{\tau}      %tau
\newcommand{\up}{\upsilon}   %upsilon
\newcommand{\f}{\phi}        %phi
\newcommand{\F}{\Phi}        %Phi
\newcommand{\x}{\chi}        %chi
\newcommand{\y}{\psi}        %psi
\newcommand{\Y}{\Psi}        %Psi
\newcommand{\w}{\omega}      %omega
\newcommand{\W}{\Omega}      %Omega
%ornaments
\newcommand{\eth}{\ensuremath{^\text{th}}}
\newcommand{\rst}{\ensuremath{^\text{st}}}
\newcommand{\ond}{\ensuremath{^\text{nd}}}
\newcommand{\ord}[1]{\ensuremath{^{(#1)}}}
\newcommand{\dg}{\ensuremath{^\dagger}}
\newcommand{\bigo}{\ensuremath{\mathcal{O}}}
\newcommand{\tl}{\ensuremath{\tilde}}
\newcommand{\ol}[1]{\ensuremath{\overline{#1}}}
\newcommand{\ul}[1]{\underline{#1}}
\newcommand{\op}[1]{\ensuremath{\hat{#1}}}
\newcommand{\ot}{\ensuremath{\otimes}}
\newcommand{\wg}{\ensuremath{\wedge}}
%text
\newcommand{\tr}{\ensuremath{\hspace{1pt}\mathrm{tr}\hspace{1pt}}}
\newcommand{\Alt}{\ensuremath{\mathrm{Alt}}}
\newcommand{\sgn}{\ensuremath{\mathrm{sgn}}}
\newcommand{\occ}{\ensuremath{\mathrm{occ}}}
\newcommand{\vir}{\ensuremath{\mathrm{vir}}}
\newcommand{\spn}{\ensuremath{\mathrm{span}}}
\newcommand{\vac}{\ensuremath{\mathrm{vac}}}
\newcommand{\bs}{\ensuremath{\text{\textbackslash}}}
\newcommand{\im}{\ensuremath{\mathrm{im}\hspace{1pt}}}
\renewcommand{\sp}{\hspace{30pt}}
%dots
\newcommand{\ld}{\ensuremath{\ldots}}
\newcommand{\cd}{\ensuremath{\cdots}}
\newcommand{\vd}{\ensuremath{\vdots}}
\newcommand{\dd}{\ensuremath{\ddots}}
\newcommand{\etc}{\ensuremath{\mathinner{\mkern-1mu\cdotp\mkern-2mu\cdotp\mkern-2mu\cdotp\mkern-1mu}}}
%fonts
\newcommand{\bmit}[1]{{\bfseries\itshape\mathversion{bold}#1}}
\newcommand{\mc}[1]{\ensuremath{\mathcal{#1}}}
\newcommand{\mb}[1]{\ensuremath{\mathbb{#1}}}
\newcommand{\mf}[1]{\ensuremath{\mathfrak{#1}}}
\newcommand{\mr}[1]{\ensuremath{\mathrm{#1}}}
\newcommand{\bo}[1]{\ensuremath{\mathbf{#1}}}
%styles
\newcommand{\ts}{\textstyle}
\newcommand{\ds}{\displaystyle}
\newcommand{\phsub}{\ensuremath{_{\phantom{p}}}}
\newcommand{\phsup}{\ensuremath{^{\phantom{p}}}}
%fractions, derivatives, parentheses, brackets, etc.
\newcommand{\pr}[1]{\ensuremath{\left(#1\right)}}
\newcommand{\brk}[1]{\ensuremath{\left[#1\right]}}
\newcommand{\fr}[2]{\ensuremath{\dfrac{#1}{#2}}}
\newcommand{\pd}[2]{\ensuremath{\frac{\partial#1}{\partial#2}}}
\newcommand{\fd}[2]{\ensuremath{\frac{d #1}{d #2}}}
\newcommand{\pt}{\ensuremath{\partial}}
\newcommand{\br}[1]{\ensuremath{\langle#1|}}
\newcommand{\kt}[1]{\ensuremath{|#1\rangle}}
\newcommand{\ip}[1]{\ensuremath{\langle#1\rangle}}
\newcommand{\NO}[1]{\ensuremath{{\bm{:}}#1{}{\bm{:}}}}
\usepackage{stackengine}
\newcommand{\GNO}[1]{\setstackgap{S}{0.7pt}\ensuremath{\Shortstack{\textbf{.} \textbf{.} \textbf{.}}#1\Shortstack{\textbf{.} \textbf{.} \textbf{.}}}}
\newcommand{\cmtr}[2]{\ensuremath{[\cdot,#2]^{#1}}}
\newcommand{\cmtl}[2]{\ensuremath{[#2,\cdot]^{#1}}}
\newcommand{\floor}[1]{\ensuremath{\left\lfloor#1\right\rfloor}}
\newcommand{\ceil}[1]{\ensuremath{\left\lceil#1\right\rceil}}
%structures
\newcommand{\eqn}[1]{(\ref{#1})}
\newcommand{\ma}[1]{\ensuremath{\begin{bmatrix}#1\end{bmatrix}}}
\newcommand{\ar}[1]{\ensuremath{\begin{matrix}#1\end{matrix}}}
\newcommand{\miniar}[1]{\ensuremath{\begin{smallmatrix}#1\end{smallmatrix}}}

\title{Programming Project 3: RHF\\
\textit{Restricted Hartree-Fock}}
\author{}
\date{}
\begin{document}

\maketitle
\vspace{-2cm}

\subsection*{Extra Files}
\begin{tabular}{p{0.25\textwidth}@{}p{0.75\textwidth}}
  \ul{file name} & \ul{description} \\
  \linl{project3_input.dat}
  & shows how to get \textsc{Psi4}'s one- and two-electron integrals as \linl{numpy.array}s\\
\end{tabular}


\subsection*{Equations}
\bmit{Integrals.}
Let $\{\x_1,\cd,\x_m\}$ be spatial AO basis functions.
\begin{gather}
\begin{aligned}
\label{oei}
  S_{\mu\nu}
=&\
  \ip{\x_{\mu}|\x_{\nu}}
&&
  T_{\mu\nu}
=
  -\tfrac{1}{2}\ip{\x_{\mu}|\nabla_1^2|\x_{\nu}}
&&
  V_{\mu\nu}
=
  \sum_A\ip{\x_{\mu}|\fr{Z_A}{|\bo{r}_1-\bo{R}_A|}|\x_{\nu}}
\end{aligned}\\
\begin{aligned}
\label{tei}
  (\x_\mu\x_\nu|\x_\rho\x_\si)
=
  \ip{\x_\mu\x_\rho|\x_\nu\x_\si}
=
  \ip{\x_\mu(1)\x_\rho(2)|\fr{1}{r_{12}}|\x_\nu(1)\x_\si(2)}
\end{aligned}
\end{gather}

\noindent
\bmit{Working equations.}
\begin{gather}
\begin{aligned}
\label{core-hamiltonian-and-density}
  h_{\mu\nu}
=&\
  T_{\mu\nu}+V_{\mu\nu}
\sp
  D_{\mu\nu}
=
  \sum_{i=1}^{n_\text{occ}}
  C_{\mu i} C_{\nu i}^*
\end{aligned}\\
\begin{aligned}
\label{fock-matrix}
  f_{\mu\nu}
=&\
  h_{\mu\nu}
+
  v_{\mu\nu}
\sp
  v_{\mu\nu}
=
  \sum_{\rho\si}
  (2\ip{\x_\mu\x_\rho|\x_\nu\x_\si} - \ip{\x_\mu\x_\rho|\x_\si\x_\nu})
  D_{\si\rho}
\end{aligned}\\
\begin{aligned}
\label{energy}
  E
=&\
  E_e
+
  V_\mathrm{Nu}
&&\sp
  E_e
=
  \sum_{\mu\nu}
  (h_{\mu\nu}+\tfrac{1}{2}v_{\mu\nu})D_{\nu\mu}
\end{aligned}\\
\begin{aligned}
\label{orthogonalizer}
&
  \bo{\tl{f}}\bo{\tl{C}}
=
  \bo{\tl{C}}\bm\ev
&&\sp
  \bo{\tl{f}}
=
  \bo{X}\bo{f}\bo{X}
&&\sp
  \bo{C}
=
  \bo{X}
  \bo{\tl{C}}
&&\sp
  \bo{X}
=
  \bo{S}^{-\frac{1}{2}}
\end{aligned}
\end{gather}


\subsection*{Procedure}

\bmit{Initialization.}
\begin{enumerate}
  \item Read in nuclear repulsion energy ($V_\mathrm{Nu}$) from the \linl{Molecule} object and integrals from \linl{MintsHelper}
  \item Form orthogonalizer $\bo{X}$ (eq. \ref{orthogonalizer})
\end{enumerate}

\noindent
\bmit{Iteration to self-consistency.}
\begin{enumerate}
  \item Set $D_{\mu\nu}=0$ as starting guess\footnote{This is the ``core'' guess, because $f_{\mu\nu}$ becomes the core Hamiltonian $h_{\mu\nu}$}
  \item\label{loop} Build Fock matrix (eq.~\ref{fock-matrix})
  \item Transform $\bo{f}\mapsto\bo{\tl{f}}$ to orthogonalized AO basis (eq.~\ref{orthogonalizer})
  \item Diagonalize $\bo{\tl{f}}$, yielding orbital energies $\ev_p$ and MO coefficients $\tl{C}_{\mu p}$ (eq.~\ref{orthogonalizer})
  \item Backtransform $\bo{\tl{C}}\mapsto\bo{C}$ to original AO basis (eq.~\ref{orthogonalizer})
  \item Build density matrix $\bo{D}$ (eq.~\ref{core-hamiltonian-and-density})
  \item Evaluate energy (eq.~\ref{energy}).
  \item If the energy is not converged, return to step \ref{loop}.
\end{enumerate}



\end{document}